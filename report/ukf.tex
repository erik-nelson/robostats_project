\section{State Estimation}
\label{sec:state_estimation}

State estimation addresses the problem of determining the robot's pose in the environment given noisy sensor observations.
This state estimation pipeline's pose output is a superior alternative to directly feeding in exteroceptive sensor observations to the controller and planner.
In this section, we present an Unscented Kalman Filter for fusing inertial measurement unit (IMU) and localization observations for a 2D robot.

\subsection{Process Model}
The vehicle state $\mbf{x}$ consists of the global position $\mbf{p} = \left[ p_x \ p_y \right]^T$, global heading angle $\theta$, global velocity $\mbf{v} = \left[ v_x \ v_y \right]^T$, IMU angular velocity bias in the $z$-direction $b_\omega$ and IMU linear acceleration biases in the $x$- and $y$-directions $\mbf{b}_a = \left[ b_{ax} \ b_{ay} \right]^T$.
%
\begin{align}
  \dot{\mbf{p}} &= \mbf{v} \\
  \dot{\theta} &= \omega - b_\omega - n_\omega \\
 \dot{\mbf{v}} &= \mbf{C}(\theta) \left( \mbf{a} - \mbf{b}_a - \mbf{n}_a \right) \label{eq:vdot} \\
\dot{b}_\omega &= n_{b\omega} \label{eq:bomegadot} \\
  \dot{\mbf{b}}_a &= \mbf{n}_{ba} \label{eq:badot}
\end{align}
%
The IMU measurements are comprised of the $z$-direction angular velocity $\omega$ as well as the body frame $x$- and $y$-direction linear accelerations $\mbf{a} = \left[ a_x \ a_y \right]^T$.
Both measurements are modelled as being corrupted by additive Gaussian white noise and a random walk bias driven by Gaussian white noise \eqref{eq:bomegadot}, \eqref{eq:badot}.
\begin{align}
  \mbf{n} &= \bbm \mbf{n}_a \\ n_\omega \\ \mbf{n}_{ba} \\ n_{b\omega} \ebm \sim \mathcal{N}(\mbf{0}, \mbf{Q}) \label{eq:processnoise} \\
  \mbf{Q} &= \text{diag}\{ \sigma_a^2, \sigma_a^2, \sigma_\omega^2, \sigma_{ba}^2, \sigma_{ba}^2, \sigma_{b\omega}^2 \} \label{eq:Q}
\end{align}
Since the IMU measurements are in the body frame, we use a 2$\times$2 rotation matrix $\mbf{C}(\theta)$ to rotate them into the global reference frame \eqref{eq:vdot}.
Each component of the noise vector in \eqref{eq:processnoise} is assumed to be independent, and their corresponding covariance sigma values are chosen by offline sensor characterization tests \eqref{eq:Q}.

\subsection{Correction Model}
The laser scans are used to construct a map of the environment, and a grid-based localization algorithm is used to compute the global pose (position and heading) of the vehicle with respect to the map.
The sensor model for the localization algorithm is given by
\begin{align}
  \mbf{z} &= \bbm \mbf{p} \\ \theta \ebm + \bbm \mbf{C}(\theta) & \mbf{0} \\ \mbf{0}^T & 1 \ebm \mbf{w} \label{eq:observationmodel} \\
  \mbf{w} &\sim \mathcal{N}(\mbf{0}, \mbf{R} ) \label{eq:obsnoise}
\end{align}
The observation noise covariance matrix $\mbf{R}$ \eqref{eq:obsnoise} is computed by fitting a multivariate Gaussian to the posterior probability grid of the localization algorithm.
Because this covariance is expressed in the scanner frame (assumed to be coincident with the body frame), we rotate its associated noise vector $\mbf{w}$ into the world frame in \eqref{eq:observationmodel}.

\subsection{Unscented Kalman Filter}
The nonlinearities introduced by the rotation in the process and correction models motivated the choice of the Unscented Kalman Filter in this project.
For brevity, we omit the complete presentation of all the steps involved in the UKF (we refer the reader to \cite{Julier97_SPIE}).
We apply the process model and correction model equations when the relevant measurement is received by the state estimator.
Outlier exteroceptive observations are rejected by a chi-squared innovation gate.
