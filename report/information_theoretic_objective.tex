\section{Information-Theoretic Objective}
\label{sec:information_theoretic_objective}

The goal of active perception exploration is to find a dynamically feasible
sequential set of actions over a time interval, $\tau := t+1,\dots,t+T$, which
enable the robot to explore its environment. We use the critera that an
explorative action is one which allows the robot to position itself in locations
that generate observations which are informative to the robot's map. Under this
criteria, choosing the optimal exploration action will maximally reduce the
uncertainty in the robot's map. An \textit{action} can be defined as a discrete sequence of
states, $x_{\tau} = \left[x_{t+1},\dots,x_{t+T}\right]$. While executing an action,
the robot will obtain a set of measurements $z_{\tau}(x_{\tau}) =
\left[z_{t+1}(x_{t+1}),\dots,z_{t+T}(x_{t+T})\right]$ by sensing from the states
$x_{\tau}$. $z_{\tau}(x_{\tau})$ is modeled as a random variable whose distribution
is parameterized by a deterministic action, $x_{\tau}$. In our formulation, we
choose to select actions from a library of motion primitives, $\mc{X}$, which
are generated by a planner. Under these notations, an explorative planner must
determine $x_{\tau}^{*}$: the action that visits locations which allow the robot
to obtain the set of measurements which maximally reduce uncertainty in its
current map.

To determine $x_{\tau}^{*}$, we follow Charrow et al. \cite{charrow15} and maximize the
Cauchy-Schwarz Quadratic Mutual Information (CSQMI) rate between the current map and the
expectation of future measurements collected along an action, $\text{I}_{\text{CS}}[m; z_{\tau} \vert
x_{\tau}]$.
%
\begin{align}
  \begin{split}
    x_{\tau}^{*}
    &=
    \argmax_{x_{\tau} \in \mc{X}^{T}}
    \frac{\text{I}_{\text{CS}}
      \left[
        m
        ;
        z_{\tau}(x_{\tau})
      \right]
    }
    {R\left(x_{\tau}\right)} \\
    \label{eq:mutual_information}
  \end{split}
\end{align}
%
Here, $R : \mbb{R}^{\vert x_{\tau} \vert} \rightarrow \mbb{R}^{+}$ computes the
estimated time required to complete action $x_{\tau}$. We choose to maximize
CSQMI as opposed to more common metrics, such as Shannon's mutual information
(MI), due to the property that CSQMI can be computed exactly
in $\mc{O}(N^{2})$, and closely in $\mc{O}(N)$, when $N$ is the number of occupancy
grid cells intersected by all rays in $z_{\tau}$. In contrast,
MI requires averaging many $\mc{O}(N^{2})$ Monte Carlo
samples. The full MI and CSQMI solutions are remarkably similar when computed
over an occupancy grid map, further motivating CSQMI's use. CSQMI between the
map and collection of measurements from $\tau$ is given in
Eq.~\eqref{eq:csqmi}.
%
\begin{align}
  \begin{split}
    &\text{I}_{\text{CS}}
    \left[
      m;
      z_{\tau}
    \right]
    = \\
    &\ -\log
    \frac
    {
      (
      \sum
      \int
      p(m, z_{\tau})
      p(m)
      p(z_{\tau})
      dz_{\tau}
      )^{2}
    }
    {
      \sum
      \int
      p^{2}(m, z_{\tau})
      dz_{\tau}
      \sum
      \int
      p^{2}(m)
      m^{2}(z_{\tau})dz_{\tau}
    }
    \label{eq:csqmi}
  \end{split}
\end{align}
%
where sums are over possible enumerations of $m$, and integrals are over the
space of measurements that can be observed during $\tau$. CSQMI is non-negative,
and maps two random variables to a real valued quantity
which represents the information that one learns about each variable by learning
the other. CSQMI is equal to zero i.f.f. its arguments are independent.

We represent measurements as $B$-tuple random variables, such that a measurement
$z_{k}^{b}$ captured at time $k \in \tau$ contains $B$ beams. Most occupancy grid
measurement models assume that cells not intersected
by a laser beam are not updated in the map. Let $c$ be the set of cells
intersected by laser beam $z_{k}^{b}$, and let $C = \vert c \vert$. Then
$\text{I}_{\text{CS}}[m; z_{k}^{b}] =
\text{I}_{\text{CS}}[c; z_{k}^{b}]$. Charrow et al. \cite{charrow15} derive a closed form
solution to the CSQMI between
a single laser beam and the robot's map.
%
\begin{align}
  \begin{split}
    \text{I}_{\text{CS}}
    [
      c;
      z_{k}^{b}
    ]
    &=
    \log
    \sum_{l=0}^{C} w_{l}
    \mc{N}(0, 2\sigma^{2}) \\
    &+ \bigg(\log
    \prod_{i=1}^{C}
    (o_{i}^{2} + (1 - o_{i})^{2}) \\
    &\ \ \ \sum_{j=0}^{C}
    \sum_{l=0}^{C}
    p(e_{j})
    p(e_{l})
    \mc{N}(\mu_{l} - \mu_{j}, 2\sigma^{2}) \bigg)\\
    &-2\log
    \sum_{j=0}^{C}
    \sum_{l=0}^{C}
    p(e_{j}) w_{l}
    \mc{N}(\mu_{l} - \mu_{j}, 2\sigma^{2})
    \label{eq:single_beam}
  \end{split}
\end{align}
%
Here, $p(e_j)$ is the probability that the $j$th cell in the raycast $c$ is the
first occupied cell, and $o_i = p(c^{i} = 1)$, the probability that $i$th cell
in the raycast is occupied. Weights, $w_{l \in [1, C]}$ can be pre-computed over
the raycast to avoid duplication, and are evaluated as
%
\begin{align}
  \begin{split}
    w_{l} &=
    p^{2}(e_{l})
    \prod_{j=l+1}^{C}
    (o_{j}^{2} + (1 - o_{j})^{2})
  \end{split}
\end{align}
%
By assuming that the resolution of the occupancy grid is greater than the
variance of the range sensor, which is typically the case in occupancy grid
mapping scenarios, and by assuming that Gaussians approach zero with growth in $\vert \mu_{l} -
\mu_{j} \vert$, one may assert that the inner sum in the double sum terms is
only non-zero when $l$ is close to $j$. In other words, the $\mc{O}(C^2)$ double
sum terms can be reduced to $\mc{O}(C)$ complexity by assuming
%
\begin{align}
  \begin{split}
    \sum_{j=0}^{C}
    \sum_{l=0}^{C}
    \alpha_{j, l}
    &\approx
    \sum_{j=0}^{C}
    \sum_{l=j-\delta}^{j+\delta}
    \alpha_{j, l} \\
    \sum_{j=0}^{C}
    \sum_{l=0}^{C}
    \beta_{j, l}
    &\approx
    \sum_{j=0}^{C}
    \sum_{l=j-\delta}^{j+\delta}
    \beta_{j, l}\\
  \end{split}
\end{align}
%
where $\delta \ll C$, and $\alpha_{j,l}$ and $\beta_{j, l}$ are the weighted Gaussian terms inside
of the double sums in Eq.~\eqref{eq:single_beam}. Our simulator and robot's
laser scanners have $\sigma \approx 0.1$ cm, and use occupancy grid sizes of
$10.0$ cm, allowing us to choose $\delta = 1$ with minimal loss of information.

