
% Single column for now
\documentclass[conference]{IEEEtran}
%\documentclass{article}

\usepackage{graphicx}
\graphicspath{{figures/}}
\DeclareGraphicsExtensions{.pdf,.jpeg,.png,.eps}
\usepackage[cmex10]{amsmath}
\usepackage{amsfonts}
\usepackage{amssymb}
\usepackage{algorithmic}
\usepackage{array}
\usepackage{mdwmath}
\usepackage{mdwtab}
\usepackage{eqparbox}
\usepackage{caption}
\usepackage{subcaption}
%\usepackage[tight,footnotesize]{subfigure}
%\usepackage[caption=false,font=footnotesize]{subfig}
\usepackage{fixltx2e}
\usepackage{stfloats}
\usepackage{url}
\usepackage[a4paper,bindingoffset=0.2in,left=1in,right=1in,top=1in,bottom=1in,footskip=.25in]{geometry}

\bibliographystyle{plain}
\DeclareGraphicsExtensions{.pdf,.jpeg,.png,.eps}
\graphicspath{{figures/}}

\input{notation}

\begin{document}

\title{Sparse Planning Graphs for \\ Information Driven Exploration}

% This block only works in IEEE mode
\author{\IEEEauthorblockN{1,2,3}
 \IEEEauthorblockA{The Robotics Institute\\
   Carnegie Mellon University\\
   Pittsburgh, PA 15217\\
Email: \{1,2,3\}@cmu.edu}}

\date{}

%\author{
  %Erik Nelson \qquad Vishnu Desaraju \qquad John Yao \\
  %\\
  %\small The Robotics Institute \\
  %\small Carnegie Mellon University \\
  %\small Pittsburgh, PA 15217 \\
%\small \{\texttt{enelson},\ \texttt{rajeswar},\ \texttt{johnyao}\}\texttt{@cmu.edu}}

\maketitle

%\begin{abstract}
%  \boldmath
%  \dots
%\end{abstract}

%\section{Progress}
%We plan to turn the results of our project into a conference paper, which we will submit to a January or February conference. Because of this, the following report will read like a draft of the formulation section in a conference paper. We have developed the basic formulation and problem statement, but in this report we have not formally described the novel approach we will use to speed up mutual information computation. In addition, in this draft of the paper we do not yet discuss integration of the RRT planner with mutual information. We have strategies for both of these aspect of our project, which are fully implemented in C++, and are integrated with our lab's SLAM system. In addition, we have our RRT planner working in both simulation and onboard our quadrotors. We will detail our efficient mutual information exploration strategy more thoroughly in our three-quarters and final reports. The introduction and a draft of the formulation for our conference paper follow.

\section{Introduction}
\label{section:introduction}

Exploration is a key capability that enables robotic vehicles to operate in unknown environments.
In general, the shortest trajectory over which a robot can greedily explore an environment is the trajectory formed by choosing actions which maximally reduce uncertainty in the environmental belief representation, or map.
One strategy to accomplish this is to choose control actions that maximize the expected mutual information between the robot's map at a future timestep and the robot's current map.
This solution is one variant of a broad category of exploration strategies known as active Simultaneous Localization and Mapping (SLAM) \cite{thrun2005probabilistic}.

State of the art active SLAM implementations generally only compute actions over a one-step planning horizon due to the extremely high computational cost of determining expected mutual information over all potential future locations \cite{bourgault2002information}, \cite{stachniss2005information}.
However, plans over much longer horizons can be generated if these expected mutual information values can be computed once, cached, and updated efficiently based on new information.
An even more efficient approach would utilize sparse planning graphs to limit expected mutual information calculations to a select few feasible future locations in the map.
Rapidly Exploring Random Trees (RRT) and lattice graphs are two examples of sparse planning graphs which decompose the reachable space into a sparse set of goal states based on a set of motion primitives for planning purposes.

This project seeks to develop a recursive formulation for efficiently and intelligently updating expected mutual information over a finite horizon as the robot moves through an unknown environment and builds a map from sensor observations.

% TODO: say what each of the sections in this report will be about? e.g. In Section \ref{section:occupancy_grid_mapping} we provide a brief overview of how a map is constructed from sensor observations, Section \ref ... etc

\section{Occupancy Grid Mapping}
\label{section:occupancy_grid_mapping}

We represent the map as an occupancy grid, which consists of a set of cells: $m = \{m^{i}\}_{i=1}^{N}$.
The probability that an individual cell is occupied is given by $p\left(m^{i} \ \vert \ x_{1:t}, z_{1:t}\right)$, where $x_{1:t}$ denotes the history of states of the vehicle, and $z_{1:t}$ denotes the history of range observations accumulated by the vehicle.
We assume that cell occupancy probabilities are independent of one another: $p\left(m \ \vert \ x_{1:t}, z_{1:t}\right) = \prod_{i} p\left(m^{i} \ \vert \ x_{1:t}, z_{1:t}\right)$.
For notational simplicity we write the map conditioned on random variables $x_{1:t}$ and $z_{1:t}$ as $p_{t}\left(m\right) := p\left(m \ \vert \ x_{1:t}, z_{1:t}\right)$.
Additionally, unobserved grid cells are assigned a uniform prior of being occupied.

We represent the occupancy status of grid cell $m^i$ at time $t$ with a log odds expression
\begin{align}
  l_t &:= \log \frac{p \left( m^i \ \vert \ z_{1:t} \right)}{p \left( \bar{m}^i \ \vert \ z_{1:t} \right)}
\end{align}
, where $\bar{m}^i$ denotes the probability that $m^i$ is unoccupied.
When a new observation $z_t$ is obtained, the log odds update is given by
\begin{align}
  l_t &= l_{t-1} + \log \frac{p\left( m^i \ \vert \ z_t \right)}{p \left( m^i \right)} - \log \frac{p\left( \bar{m}^i \ \vert \ z_t \right)}{p \left(\bar{m}^i \right)}
\end{align}
where the last two terms represent the inverse sensor model.

\section{Occupancy Grid Mapping}
\label{sec:occupancy_grid_mapping}

In order to develop an information-theoretic reward surface, we model the
robot's map as a binary random variable. We therefore discretize the space, and
represent the map as an occupancy grid - a common environmental representation
for robotic mapping.

As a basis for the core formulation in the following sections, we provide a brief overview of occupancy grid mapping. Occupancy grids are a common and useful probabilistic model for representing an environment.
We represent the map as an occupancy grid, which consists of a set of cells: $m = \{m^{i}\}_{i=1}^{N}$.
The probability that an individual cell is occupied is given by $p\left(m^{i} \ \vert \ x_{1:t}, z_{1:t}\right)$, where $x_{1:t}$ denotes the history of states of the vehicle, and $z_{1:t}$ denotes the history of range observations accumulated by the vehicle.
We assume that cell occupancy probabilities are independent of one another: $p\left(m \ \vert \ x_{1:t}, z_{1:t}\right) = \prod_{i} p\left(m^{i} \ \vert \ x_{1:t}, z_{1:t}\right)$.
For notational simplicity we write the map conditioned on random variables $x_{1:t}$ and $z_{1:t}$ as $p_{t}\left(m\right) := p\left(m \ \vert \ x_{1:t}, z_{1:t}\right)$.
Additionally, unobserved grid cells are assigned a uniform prior of being occupied.

We represent the occupancy status of grid cell $m^i$ at time $t$ with a log odds expression
\begin{align}
  l_t &:= \log \frac{p \left( m^i \ \vert \ z_{1:t} \right)}{p \left( \bar{m}^i \ \vert \ z_{1:t} \right)}
\end{align}
where $\bar{m}^i$ denotes the probability that $m^i$ is unoccupied.
When a new observation $z_t$ is obtained, the log odds update is given by
\begin{align}
  l_t &= l_{t-1} + \log \frac{p\left( m^i \ \vert \ z_t \right)}{p \left( m^i \right)} - \log \frac{p\left( \bar{m}^i \ \vert \ z_t \right)}{p \left(\bar{m}^i \right)}
  \label{eq:logodds_update}
\end{align}
where the last two terms represent the inverse sensor model.

\section{Explorative Information Cost Function}
\label{section:explorative_information_cost_function}

In this section, we will introduce an information-theoretic formulation for planning explorative paths. This formulation does not retain or propagate information from previous iterations, and is therefore slow.

The goal of exploration is to find a dynamically feasible sequential set of motions chosen from a library of motion primitives, $\mc{X}$, over a time interval, $\tau := t+1 : t+T$, which enable the robot to explore its environment. We use the criteria that an explorative action is one which allows the robot to position itself in locations that generate observations which are informative to the robot's map. By this criteria, choosing the optimal exploration action will maximize an information metric over the robot's \textit{future map}. In this context, a \textit{future map} is an updated version of the robot's current map, which has used Eq.~\eqref{eq:logodds_update} to incorporate all measurements gathered while executing an action. An \textit{action} can be defined as a discrete sequence of states, $x_{\tau} = \left[x_{t+1},\dots,x_{t+T}\right]$. While executing an action, the robot will obtain a set of measurements $z_{\tau}(x_{\tau}) = \left[z_{t+1}(x_{t+1}),\dots,z_{t+T}(x_{t+T})\right]$ by sensing from the states $x_{\tau}$. $z_{\tau}(x_{\tau})$ is modeled as a random variable whose distribution is parameterized by a deterministic action, $x_{\tau}$, generated by a planner. Under this notation, an explorative planner must determine $x_{\tau}^{*}$: the action that visits locations which allow the robot to obtain the set of measurements which are most informative to the current map. When integrated with Eq.~\eqref{eq:logodds_update}, these measurements will maximize an information-theoretic cost function over the map. We choose to maximize Shannon Mutual Information (MI) rate between the current map, $m$, and the measurements $z_{\tau}$ gathered along $x_{\tau}$.
%
\begin{align}
  \begin{split}
    x_{\tau}^{*}
    &=
    \argmax_{x_{\tau} \in \mc{X}^{T}}
    \frac{\text{I}_{\text{MI}}
      \left[
        m
        ;
        z_{\tau}(x_{\tau})
      \right]
    }
    {R\left(x_{\tau}\right)}
    =
    \argmax_{x_{\tau} \in \mc{X}^{T}}
    \frac{
      \text{H}
      \left[
        m
      \right]
      -
      \text{H}
      \left[
        m
        \ \vert \
        z_{\tau}(x_{\tau})
      \right]
    }
    {R\left(x_{\tau}\right)}
    \label{eq:mutual_information}
  \end{split}
\end{align}
%
where $\text{I}_{\text{MI}}$ is the Shannon Mutual Information, and $R: \mc{X}^{T} \rightarrow \mbb{R}^{+}$ is a function which returns the time required to execute an action. In contrast to MI, MI rate is used so that actions with different execution times can be compared under a common metric.

Fortunately, $\text{H}\left[m\right]$ depends only on the current map, and can be computed once for all enumerations of $x_{\tau}$ under consideration. Several additional assumptions can be made to simplify the optimization of $x_{\tau}^{*}$. Due to cell independence in the occupancy grid formulation, the joint conditional entropy over the map can be expressed as a sum of individual cell conditional entropies. Additionally, let $\mc{C}$ to be the set of cells in the map that beams in $z_{\tau}$ pass through. Given the inverse sensor model (shown in the last two terms of Eq.~\eqref{eq:logodds_update}), cells $c \notin \mc{C}$ are not updated by $z_{\tau}$ and therefore do not contribute to the map's conditional entropy. Using these assumptions, we may write the entropy of the map conditioned on $z_{\tau}$ as
%
\begin{align}
  \begin{split}
    \text{H}
    \left[
      m
      \ \vert \
      z_{\tau}
    \right]
    &=
    \sum_{c \in \mc{C}}
    \text{H}
    \left[
      c
      \ \vert \
      z_{\tau}
    \right]
    \\
    &=
    \sum_{c \in \mc{C}}
    \expect_{c, z_{\tau}}
    \left[
      -\log
      p
      \left(
      c
      \ \vert \
      z_{\tau}
      \right)
    \right]
    \\
    &=
    -
    \int_{z_{\tau}}
    p
    \left(
    z_{\tau}
    \right)
    \sum_{c \in \mc{C}}
    o
    \left(
    c
    \ \vert \
    z_{\tau}
    \right)
    d z_{\tau}
    \label{eq:conditional_entropy}
  \end{split}
\end{align}
%
where
\begin{align}
  \begin{split}
    o(c \ \vert \ z_{\tau})
    &=
    p(c \ \vert \ z_{\tau})
    \log
    p(c \ \vert \ z_{\tau})
    +
    \left(
    1
    -
    p(c \ \vert \ z_{\tau})
    \right)
    \log
    \left(
    1
    -
    p(c \ \vert \ z_{\tau})
    \right)
  \end{split}
\end{align}
%
Computing this term requires integration over the space of measurements, $z_{\tau}$, which is intractable for an online planner. Instead, we approximate Eq.~\eqref{eq:conditional_entropy} with $N$ Monte Carlo samples, $\{z_{\tau}^{i}\}_{i=1}^{N}$, drawn from the distribution $p(z_{\tau})$.
%
\begin{align}
  \begin{split}
    \text{H}
    \left[
      m
      \ \vert \
      z_{\tau}
    \right]
    &\approx
    -\frac{1}{N}
    \sum_{i=1}^{N}
    \sum_{c \in \mc{C}}
    o
    \left(
    c
    \ \vert \
    z_{\tau}^{i}
    \right)
    \label{eq:monte_carlo}
  \end{split}
\end{align}
%
Substituting Eq.~\eqref{eq:monte_carlo} into Eq.~\eqref{eq:mutual_information} yields an information-theoretic cost function, which, when maximized, computes an explorative action.
%
\begin{align}
  \begin{split}
    x_{\tau}^{*}
    &=
    \argmax_{x_{\tau} \in \mc{X}^{T}}
    \frac{1}{R\left(x_{\tau}\right)}
    \left(
      \text{H}
      \left[
        m
      \right]
      +
      \sum_{i=1}^{N}
      \sum_{c \in \mc{C}}
      o
      \left(
      c
      \ \vert
      \ z_{\tau}^{i}(x_{\tau})
      \right)
      \right)
      \label{eq:cost_function}
  \end{split}
\end{align}
%
This cost function depends on the ability to sample from $p(z_{\tau})$, necessitating a reliable measurement model from which to generate representative measurements.

\section{Measurement Model}
\label{sec:measurement_model}

The CSQMI of a single beam (Eq.~\eqref{eq:single_beam}) depends on the ability
to estimate $p(e_j)$, the probability that the $j$th cell in a raycast is
the first occupied grid cell. Rather than building a measurement model using the
maximum likelihood estimate of $z_{j}$ like other similar works \cite{charrow15,
thrun2005probabilistic, julian2013mutual}, we compute a discrete distribution over the
raycast, where each cell value directly approximates the probability cell $c^i$ is the
first occupied cell in $c$, $p(e_j = c^i) \ \forall c^{i} \in c$ using the
continuous occupancy values in $c$.

To calculate this distribution, we use a generalization of the geometric
distribution, which is commonly used to determine the probability distribution
over the number of Bernoulli trials necessary to obtain one success, supported
on the set $\mbb{N}^{+}$. Since each occupancy grid cell in a raycast contains
continuous probababilities $\in [0, 1]$ of the cell's occupancy, we build on the
geometric distribution to support independent, but not identically distributed
(i.n.i.d) random variables. The probability that a cell $c^{i}$ is the first
cell that terminates the raycast is then the probability that no previous cells
in the raycast did.
%
\begin{align}
  \begin{split}
    p(e_j = c^{i})
    &=
    o^{i}
    \prod_{j=1}^{i-1}
    (1 - o^{j})
    \label{eq:geometric_brute}
  \end{split}
\end{align}
%
Although Eq.~\eqref{eq:geometric_brute} requires $\mc{O}(C^2)$ to be
computed for all $c^{i} \in c$, we
were able to derive an efficient recursive formula to compute the distribution $p(e_j)$,
which runs in $\mc{O}(C)$.
%
\begin{align}
  \begin{split}
    p(e_j = c^{i})
    &=
    o^{i}
    \prod_{j=1}^{i-1}
    (1 - o^{j}) \\
    &=
    o^{i}
    \left(
    \frac
    {o^{i-1}}
    {o^{i-1}}
    \right)
    (1 - o^{i-1})
    \prod_{j=1}^{i-2}
    (1 - o^{j}) \\
    &=
    o^{i}
    \frac
    {
      1 - o^{i-1}
    }
    {
      o^{i-1}
    }
    p(e_j = c^{i-1}) \\
    &=
    o^{i}
    \left(
    \frac{1}{o^{i-1}}
      -
      1
    \right)
    p(e_j = c^{i-1})
  \end{split}
\end{align}
%
The distribution $p(e_j)$ is depicted over a one-dimensional raycast in Fig.~\ref{fig:measurement_model}.

\begin{figure}
  \centering
  \includegraphics[width=0.5\textwidth]{meas_model.pdf}
  \caption{The distribution $p(e_j)$ over a depicted 1-dimensional map with $p_{e} = 0.01$, $p_{o} = 0.92$, $p_{u} = 0.05$. \label{fig:measurement_model}}
\end{figure}


\section{Closed Loop RRTs}
\label{sec:planner}

To make use of this information cost function to guide exploration, we consider a sampling-based planning approach that can evaluate the predicted information gain throughout the environment. In addition, we wish to use the occupancy grid that is being updated online (as described in Sect.~\ref{section:occupancy_grid_mapping}) to guide the vehicle around obstacles in the environment. The rapidly-exploring random tree (RRT) algorithm is well suited to planning paths through these types of large environments, and works as follows.
The planner starts from the vehicle's current state and samples a point $x$ in the environment. Using the occupancy grid, we can reject samples that lie in cells with a sufficiently high probability of being occupied. If the sample is valid, we find the closest node in the tree of paths (initially just the vehicle state), where closeness is measured in terms of Euclidean distance, and add a new edge to the tree connecting the sample point to the nearest node.
Then a new sample is drawn and the process repeats to grow a tree of path segments through the environment. This tree-growing process terminates after a specified time, and the minimum cost path is returned.

Figure~\ref{fig:} shows snapshots of the system planning through the environment while updating the occupancy grid. The edges in the tree are smooth since they are generated by forward simulating the closed-loop vehicle dynamics toward the sample point, resulting in a variant of the RRT algorithm known as Closed-loop RRT (CL-RRT)~\cite{Kuwata09_TCST}. This approach is traditionally used to ensure dynamic feasibility and dense collision checking. However, the forward simulation also means we have full state information for the system at the end of each segment. This allows us to evaluate the predicted information gain at that point and assign a corresponding cost to candidate trajectory.

To change CL-RRT from a goal-directed planner to an exploration-driven planner, we first define the sampling distribution to be a Gaussian centered about the root of the tree, with no bias toward any direction (unlike standard sampling-based planners that will sample the goal some small probability). We also define the cost of each branch segment to just be the information metric computed at its endpoint (as opposed to a more traditional setup where the cost is the total distance traveled from the root plus a cost to go based on an admissible heuristic, such as Euclidean distance to a goal). Finally, since there is no goal to guide the selection of the best branch from the tree, we simply select the branch with the minimum cost endpoint in the entire tree. This enables the planner to compute paths that aim to maximize the predicted information gain.

Define $\mathcal{X}_\text{free}$ to be the space spanned by the unoccupied cells in the occupancy grid.

\begin{algorithm}
\caption{CL-RRT: Tree Expansion}
\label{alg:clrrt_expansion}
\begin{algorithmic}[1]
\State Sample point $x_s$ from the environment
\State Select min-cost node from $n$ nearest in tree
\State $k \gets 0$
\State $\hat{x}(t+k) \gets $ last state at n
\While{$\hat{x}(t+k) \in \mathcal{X}_\text{free}(t+k)$ and $\hat{x}(t+k) \neq x_s$}
	\State Compute control input to drive system to $x_s$
	\State Forward simulate system dynamics
	\State Compute next state $\hat{x}(t+k+1)$ from propagation model
	\State $k \gets k+1$
\EndWhile
$N \gets r_final$
\For{each feasible node $N$ produced}
	\State Update cost estimates for $N$
	\State Add $N$ to tree
\EndFor
\end{algorithmic}
\end{algorithm}

\section{Unscented Kalman Filter}
\label{sec:unscented_kalman_filter}




\bibliography{refs}

\end{document}








%\subsection{Subsection Heading Here}
%
%\begin{figure}[!t]
%\centering
%\includegraphics[width=2.5in]{3quads.png}
%\caption{Simulation Results}
%\label{fig_sim}
%\end{figure}
%\begin{table}[!t]
%\renewcommand{\arraystretch}{1.3}
%\caption{An Example of a Table}
%\label{table_example}
%\centering
%\begin{tabular}{|c||c|}
%\hline
%One & Two\\
%\hline
%Three & Four\\
%\hline
%\end{tabular}
%\end{table}

%\appendices
%\section{Proof of the First Zonklar Equation}

%\begin{thebibliography}{1}

%\bibitem{IEEEhowto:kopka}
%H.~Kopka and P.~W. Daly, \emph{A Guide to \LaTeX}, 3rd~ed.\hskip 1em plus
%  0.5em minus 0.4em\relax Harlow, England: Addison-Wesley, 1999.
%\end{thebibliography}
