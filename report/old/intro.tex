\section{Introduction}
\label{section:introduction}

Exploration is a key capability that enables robotic vehicles to operate in unknown environments.
In general, the shortest trajectory over which a robot can greedily explore an environment is the trajectory formed by choosing actions which maximally reduce uncertainty in the environmental belief representation, or map.
One strategy to accomplish this is to choose control actions that maximize an information metric between the robot's current map and the robot's map at a future timestep.
This solution is one variant of a broad category of exploration strategies known as active Simultaneous Localization and Mapping (SLAM) \cite{thrun2005probabilistic}.

State of the art active SLAM implementations generally only compute actions over a one-step planning horizon due to the extremely high computational cost of determining expected information gain over all potential future locations \cite{bourgault2002information}, \cite{stachniss2005information}.
However, plans over much longer horizons can be generated if expected information gain values can be computed once per planning step, cached, and updated efficiently based on new information.
An even more efficient approach would utilize sparse planning graphs to limit expected information gain calculations to a select few feasible future locations in the map.
Rapidly Exploring Random Trees (RRT) and lattice graphs are two examples of sparse planning graphs which decompose the reachable space into a sparse set of goal states based on a set of motion primitives for planning purposes.

This project seeks to develop a recursive formulation for efficiently and intelligently updating expected information gain over a finite horizon as the robot moves through an unknown environment and builds a map from sensor observations.

% TODO: say what each of the sections in this report will be about? e.g. In Section \ref{section:occupancy_grid_mapping} we provide a brief overview of how a map is constructed from sensor observations, Section \ref ... etc


