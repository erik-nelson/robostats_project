\section{Motivation}

Guided by the desire to enable ultra-fast robot exploration, we plan to focus our project towards developing an information-based belief representation for rapid planning and localization. In general, the shortest trajectory over which a robot can greedily explore an environment is the trajectory formed by choosing actions which maximally reduce uncertainty in the environmental belief representation. One strategy to accomplish this is to choose control actions that maximize the expected mutual information between the robot's map at a future timestep and the robot's current map. This solution is one variant of a broad category of exploration strategies known as active Simultaneous Localization and Mapping (SLAM) \cite{thrun2005probabilistic}.

{\bf State of the art active SLAM implementations generally only compute actions over a one-step planning horizon} due to the extremely high computational cost of determining expected mutual information over all potential future locations \cite{bourgault2002information}, \cite{stachniss2005information}. {\bf However, we believe that plans over much longer horizons can be generated} if these expected mutual information values can be computed once, cached, and updated efficiently based on new information. An even more efficient approach would utilize sparse planning graphs to limit expected mutual information calculations to a select few feasible future locations. Rapidly Exploring Random Trees (RRT), and lattice graphs are two examples of sparse planning graphs which decompose the reachable space into a sparse set of goal states based on a set of motion primitives for planning purposes.

We propose to investigate mutual information propagation through sparse planning graphs. {\bf A substantial impact of this work} would be to develop a recursive formulation for efficiently and intelligently updating expected mutual information over a finite horizon as the robot navigates and obtains information through its sensors. {\bf This project is deeply intertwined with the concepts of probability and uncertainty} due to the fact that at every update step, our algorithm must predict future measurements and future maps based on current sensor models and environmental belief.
